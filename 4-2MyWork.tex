\subsubsection{More realist/handy : Adding "wash hands / take a bath" instead of Sit text}
Only if used, or if nothing better
\subsubsection{More intuitive UI : Changing the onClick logo}
URL : % http://wiki.secondlife.com/wiki/LlSetClickAction/fr

%%%%%%%%%%%

\subsection{Food system}

\subsubsection{General overview}
The implementation of a food system that allow the representation of a diabetic avatar was designed following the created model described in section~\ref{sec:mymodelSection} and was made as generic as possible so that the model can be easily enhanced or changed.

At each time, glucose level can be computed from the model according to global and local parameters as shown in figure~\ref{fig:GLparam}. These parameters are located in two scripts : StateScript and FoodObject. As described in section~\ref{}, StateScript is the main script for the avatar, controlling the glucose-insulin system and the avatar's behaviour related to diabetes such as triggering hypoglycemia or hyperglycemia symptoms. FoodObject is a generic script that describes each food item behaviour. 

According to our requirements, a minimal amount of specific features need to be implemented for each kind of food : the total amount of glucose in a food item, the time needed to reach the glucose peak (maximum level of blood glucose since the food consumption). This generic skeleton is used for each food object in the simulation. The total amount of glucose is taken from tables such as~\ref{fig:foodGlucoseTable} and \cite{nutritiondataWebsite} %http://nutritiondata.self.com/ 
when the data can't be found in other tables. 
\\
The Glycemic Index (GI) and the Glycemic Load are two measures %values
that can be found in such tables. Used by nutritionists since 1980, the glycemic index characterizes how quickly blood glucose rises after the consumption of a specific kind of food. 
Created in the 1980s, the GI is based on a reference, usually pure glucose or white bread, valued at 100 which is the highest possible GI as shown in figure~\ref{manyIG}. 30 minutes after the food consumption, a higher GI induces a higher blood glucose peak. It can also be noticed that if the GI is between medium and high, an reactive (or postprandial) hypoglycemia is observed after 2 hours, which wouldn't be observed if the consumed food had a lower GI.

If the temporal rise of blood glucose can be predicted by using GI, this index doesn't take into account the amount of ingested carbohydrates. For this purpose, the glycemic load (GL) was established to estimate how much the blood glucose will rise after eating a specific food. Knowing the food serving weight and the amount of glucose inside, this formula gives the link between the glycemic index and the glycemic load : $GL = IG * (mglucose/mfood)$
\\

\paragraph{Food object : providing a realistic experience}
\label{sec:food}
The food object is an object that represents a food item in 3D, with a script inside to make the object eatable. The food object doesn't do any calculation, but sends to the StateScript the reference values that correspond to the food. The main issue to solve with the script object was to give it a realistic behaviour. 
\\

To trigger the action of eating the food, different solutions are possible : 
\begin{enumerate}
\item The food is eaten when the avatar right-clicks on the object to access the contextual menu, then selects the "Eat" action. The addition of personalized commands in the contextual menu was needed in some other cases, and the solution to make it is explained in section~\ref{sec:menuAction}. 
\item The food is eaten when the avatar "walks" on the object, as in some video games.
\item The food item is eaten when the avatar simply clicks on it. A minor problem with this solution is that there is no way to know in advance which object is "eatable" if there's nothing to show it's clickable, like a different mouse icon. As I planned to remove the text written on the food, it will be hard to detect with which object of the kitchen interactions are available.
\end{enumerate}
Maybe explain pros and cons of each, and why I finally choose this one. %TODO : finish

\paragraph{Different animations : an animation library (generalize it for the insulin injection ?)}
\label{sec:animLib}

According to the food objects nature, the eating animation played will be different. In fact, eating a sandwich is different of eating a bit of chocolate: in the first case, the two hands may be needed, and the animation takes longer as eating a sandwich takes more time, while in the second case, only one bite is needed.

2 kinds of elementary animations were used : 
-for single item/simple bite and for plate
-for drinking

The idea was to loop them for food objects that need many bites. By combination, the simple bite animation could be used for many purposes :
-to eat a simple item, like a crisp
-combining a right hand simple bite animation with a left hand one to eat sandwiches
-adding another animation for cutting food in a plate/taking a piece of food in a plate, and then playing the simple bite animation.

For drinking, the animation was simply looped.

\paragraph{Food detruction}
\label{foodDestruction}
Food destruction needed to be considered at the very beginning, because after being called by the script, the llDie() function interrupts the execution of the script. As a consequence, the food script cannot ensure the progressive release of glucose in the avatar's blood over the time, that is supposed to last about 2 hours. 
Moreover, the destruction of the food object need to be synchronized with the end of the eating animation : if the objects dies too early, the avatar will appear as eating nothing, and if it dies too late, the food item will disappear unrealistically after the end of the eating movements, while being in the hand. If possible, the food object should die when it's close to the mouth, during the latest bite. As all food items are not using the same animation while being eaten as described in section ~\ref{secDiffAnim} , the duration of the animation also needs to be stored, as an internal value for the Food script, for each kind of food. With this information, the llSleep function can be called by the script to stop its execution during the animation, waiting for the appropriated moment to kill the object. 
Computing the appropriate moment was complicated by the rights permission that asks the user allows the food attachment to his hand, triggered after clicking on the food item. This time need to be added to the computation. Moreover, the issues explained in section~\ref{sec:animLib} about playing multiple animations or looping them need to be taken into account, and need to be added to the result. Graph of the execution ? With one or two animation shown. %The animation lenght is automatically computed ... do it ?


\paragraph{Adding an action in the contextual menu}
\label{sec:menuAction}
We can add a personalized action to be executed when selected in the contextual menu~\ref{contextMenu}. The added action will replace the original default action, a "Sit" action that makes the avatar sit on the object. It is a default action allowed for any physical object by Opensim, even if the sitting action and the sitting point (see section~\ref{sittingpoint} for details on the sitting point) aren't already defined by the user for the object. 
\\For some reasons, we may want to make another action available from the contextual menu instead of the Sit action. The number of buttons in the contextual menu is fixed, so the only way to do so it to replace the Sit button. First step is to change the action executed when clicking on the button. This can be done by scripting, calling the 
llSetSitText

\subsubsection{Orientation and position issues related to the animation's constraints}

\paragraph{A position issue: the exemple of the fishing activity}
\label{sittingpoint}

Set the fishing animation.
The sitting point has to be defined, which means that any time someone sits on the boat, the avatar will be sitting at the same point, with the same body orientation. If not defined, the sitting point can be anywhere on the boat as shown in figure~\ref{fig:goodSittingPoint, fig:wrongSittingPoint}, which means that the avatar position won't match the animation. It is an important issue because there won't be any realism in this case, and the user may even not recognize the played animation.

\paragraph{An orientation issue: the Eating example}
When eating a food item as described in section~\ref{sec:food}, the food item is put in the avatar's hand, and the corresponding animation is played. For this purpose, the food item need to be placed correctly in the hand, to make it realistic. However, as shown with the glass in figure~\ref{fig:badFoodOrientation, fig:goodFoodOrientation}, this position is often not compatible with the one originally used to display the food in world, here a glass placed upright on the table.
For this purpose, two solutions can be used : 
\begin{enumerate}
\item Changing the object's position by using the llSetPosition~\cite{llSetPositionWebsite} in a script to change the orientation of the object depending on whether it is attached to an avatar or on the table. 
\item Using a "display only" object that is also used for automatic objects creation. The idea here is, for a set of brownies for instance, to create an "aggregate" of brownies, such as 7 or 8 pieces of cake on a plate.
This solution is especially convenient for some kind of food items, as it provides a way to generate automatically food when it is used. Disadvantage : the solution doesn't work for any kind of objects : creating an aggregate of pasta plates is not a good solution because a user usually wants to eat one item of this kind of food. There is also a problem as the food won't disappear from the plate when taken and eaten, which might look weird for an aggregate with a small number of objects displayed.  
The aggregate is created by arranging several food objects in the desired display, and then linking them into a "super object", the aggregate. As shown in figure~\ref{fig:aggregate}, Opensim make the aggregate appear as a single object. Thus, normal properties of objects are usable, in particular the object's inventory. Other objects can be put in the inventory and used by the scripts, so the object we want to give to the avatar (for instance, a piece of brownie with the orientation that matches the eating animation with the brownie held in the avatar's hand) can be put in this inventory and given by the main script of the aggregate to the avatar. The attachToAvatar~\cite{llAttachToAvatarWebsite} function is called for this purpose, and gives a copy of the brownie in the aggregate's inventory to the avatar. 
\end{enumerate}

Maybe explain when each solution is better and when I also used them ?


\subsection{Energy intake}
-speed+simultaneousFood
-insulin


Computation : 
 
-Energy intake
    *generic skeleton
    *speed
    
-Realistic system :
    A) food need to disappear : 
        1)was a problem : how to add energy continuously if the object that contained the food script died ? 
        2)wait a fixed amount of time (depending on the kind of objects, how long it's supposed to take to eat it) before being destroyed.
        
    B) food animation : the avatar have to seem realistically eating the food item. Thus :
        1) different kind of animations for different kind of food. 3 main animations : eat a normal item, like a cake, drink, or eat like a sandwich.
        2) food orientation issue : it was an important issue... Use a distributor object cf "Giving objects" subsection
    
    C) supporting simultaneous food items eaten when an "active period" can overlapp.
    D) Food name written from the script