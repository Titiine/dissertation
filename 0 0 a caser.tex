% PUT IN MY WORK : It looks like a progress bar indicating the amount of left energy for the avatar (or glucose in the avatar's blood). The energy indicator consists on an dynamic webpage rendered by the web server and displayed on an object in world. 

\subsection{ERASE THIS SECTION}


\subsection{Energy intake}
-speed

After being touched by the avatar, the food object communicates to the StateScript (via llSay, see section \ref{oldref...}) the characteristic parameters for the food. When the parameters data cannot be found in the literature, the only parameter sent is the total amount of carbohydrates contained in the food item, and a default behaviour is followed. The default behaviour is based on a interpolation of a sample glucose absorption function (or a curve) that was computed from the mathematical model chosen. As shown in figure ~\ref{fig:absCurve}, if we take a step of 2 seconds, the increasing of the value is saved every 2 seconds. This increasing is then converted into a percentage of the global amount of carbohydrates as described in table~\ref{tab:percentageValues}. This list of percentage values will be the model for the carbohydrates released in the blood during digestion. 
The percentage associated to each point/step is then directly used to compute the amount of carbs, and added to the global blood glucose. 

The "peak" value corresponds at t=30 minutes (in real life) corresponds to the glycemic index, given by the tables. 
The step used can be made smaller if more precision is needed for the "digestion process".

END
%%%%%%%%%%%%%%

\iffalse

+simultaneousFood
-insulin


Computation : 
 
-Energy intake
    *generic skeleton
    *speed
    
-Realistic system :
    A) food need to disappear : 
        1)was a problem : how to add energy continuously if the object that contained the food script died ? 
        2)wait a fixed amount of time (depending on the kind of objects, how long it's supposed to take to eat it) before being destroyed.
        
    B) food animation : the avatar have to seem realistically eating the food item. Thus :
        1) different kind of animations for different kind of food. 3 main animations : eat a normal item, like a cake, drink, or eat like a sandwich.
        2) food orientation issue : it was an important issue... Use a distributor object cf "Giving objects" subsection
    
    C) supporting simultaneous food items eaten when an "active period" can overlapp.
    D) Food name written from the script
    
\fi
%%%%%%%%%%%%%%
 
Coupé le passage sur les symptomes

%%%%%%%%%%%%%%%%%%%%%%%%%%%%%%%
MIS AILLEURS !!

\iffalse

The effects of hypoglycemia were then simulated by the use of animations realized using an external software, QAvimator~\cite{qavimatorWebsite}. Other softwares can be used to create and edit animations such as Poser~\cite{PoserWebsite}, Maya~\cite{mayaWebsite} and Blender~\cite{BlenderWebsite}, an open-source alternative to Maya. Qavimator was chosen because it's light and easy to use compared to Maya and Blender. Unlike Maya and Poser, it is free.\\

Qavimator is an Qt port of Avimator~\cite{avimatorWebsite}, an open source animation editor originally created for avatar's animations in Second Life. The BVH (Biovision Hierarchy)~\cite{thingvold1999biovision} format is used for the animations in Second life and is popular in the animation community to represent movements of humanoid structures. \\

A BVH file is composed of two sections, a header and a data section. The initial pose of the skeleton is described in the header section by specifiying the position of each joint, which are the green dots visibles in figure~\ref{qavimatorSkeleton}. \\

\begin{figure}[h]
  \caption{Skeleton displayed by Qavimator~\cite{qavimatorWebsite}. Each green dot represents a joint}
  \centering
  \includegraphics[scale=0.6]{qavimatorSkeleton.png}
  \label{qavimatorSkeleton}
\end{figure}

An animation is a sequence of frames, which are key moments of the animation, like the pictures for a movie. The frames are visible in the timeline window, visible in figure~\ref{qavimatorTimelineZoom1}.
By adding a new frame in the timeline, and specifiying the new position of all the joints in the current frame, an animation can be build. The position of each joint is described by a set of rotation, position and scale parameters. Figure~\ref{qavimatorTimeline} shows the editor window with 3 frames in the timeline and all the parameters for a given point (9 parameters) in the parameter window, visible in figure~\ref{qavimatorTimelineZoom2}.

\begin{figure}[h]
  \caption{Sequence of positions by Qavimator~\cite{qavimatorWebsite}. }
  \centering
  \includegraphics[scale=0.6]{qavimatorTimeline.png}
  \label{qavimatorTimeline}
\end{figure}

Insulin animation (CITE THE IMAGES Insulin1 - 4)

Some animation used in this project were available online (created by Linda Kellie), such as the piano and the fishing animation.

FIN MIS AILLEURS !!
%%%%%%%%%%%%%%%%%%%%%%%%%%ù
\fi



%%%%%%%%%%%%%%%%%%%%%%%%%%ù
MIS AILLEURS !!
\iffalse

The main animation built for this project was the one used to inject insulin. The first frame is a movement that involves the right collar, shoulder and forearm to move the insulin carried by the right hand into a correct position to begin the injection. The second frame sets the correct position for the right thight were insulin will be injected, and change the adomen and chest position to make the avatar lean forward. Next movement is to make the avatar look at his tight, while the arm comes down to put the needle into the thigh and without moving during 10 seconds, as required for an insulin injection.

FIN MIS AILLEURS !!
%%%%%%%%%%%%%%%%%%%%%%%%%%ù
\fi


=>Talk about animation priorities ? 

