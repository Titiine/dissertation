%% EVALUATION + RESULTS (?) + PERSPECTIVES ?
%%%%%%%%%%%%%%%%%%%%%%%%%%%%%%%%%%%%
%%%%%%%%%%%%%%%%%%%%%%%%%%%%%%%%%%%%
\section{Evaluation}
%%%%%%%%%%%%%%%%%%%%%%%%%%%%%%%%%%%%
%%%%%%%%%%%%%%%%%%%%%%%%%%%%%%%%%%%%

The technical and functional evaluation of the tool are described in this section. 

The technical evaluation of the tool assesses the ability of the simulation to cope with a usage in real condition by patients
while the functional evaluation evaluates to which extend the tool meets the requirements.
The tool was tested by a limited number of persons. %donc resultats chiffrés detaillés ferait peu de sens. Mettre aussi les ages, le sexe ?

The results of the evaluation will highlight the strengths and weaknesses of the tool, in order to %prévoir
the enhancements that could be made. In fact, "travail a venir" is described with the future perspectives in the last subsection.


\subsection{Technical evaluation}
\subsubsection{Stability} 
: Not bad
\subsubsection{Responsiveness} 
: %Rapidité
OK
\subsubsection{Usability} 
: Complexité de l'interface\\

+ des efforts sur les clicks\\
- des menus superflus

\subsection{Functional evaluation}
%coller les hypotheses de l'intro


\subsection{Perspectives}
%Model changes
%Indicators displayed for the symptoms/explain what's happening
%Profil personalized with sex/gender

%technical : 
	%internet support(multiplayer) + usb pendrive
	%(instead of the text instructions,) a tutorial to explain better how to use it
	%OR
	%non playing characters to explain how to use the tool (also used for motivational issues, as discussed in introduction)
	%change the viewer UI to remove what we don't want to diplay

%%%%%%%%%%%%%%%%%%%%%%%%%%%%%%%%%%%%
%%%%%%%%%%%%%%%%%%%%%%%%%%%%%%%%%%%%
\section{Results}
%%%%%%%%%%%%%%%%%%%%%%%%%%%%%%%%%%%%
%%%%%%%%%%%%%%%%%%%%%%%%%%%%%%%%%%%%

\subsection{Simulation and education : effective (=ce qu'on peut vraiment faire) usages of our tool}
\label{sec:simulationUsages}
Montrer ce qu'il est possible de faire avec cet outil (cf l'intro qui parle de simulation theorique ou pratique)\\
Ajouter que dans notre simulation, on peut apprendre:\\
-avec la simulation, en voyant les effets on devine les machanismes au niveau du corps : savoir theorique\\
-avec les materials à lire (ex carb counting) : savoir theorique encore\\
-avec les meterial video : savoir pratique\\
-avec la simulation, cette fois pour apprendre a experimenter (insulin, etc), ex pour un patient non diabetique\\
-avec la simulation, pour s'entrainer : ex pour un patient, qui perfectionne ses techniques d'injection (avec les conseils sur les sites d'injection par ex)\\

Questions :\\
`Have you seen or played any environmental computer simulation before? '
`WHAT DO YOU THINK ABOUT 3-D COMPUTER SIMULATION AS A TOOL FOR ENVIRONMENTAL EDUCATION?'
(Source :...)% Taken from : http://www.simteach.com/slccedu07proceedings.pdf?referer=http%3A%2F%2Fscholar.google.fr%2Fscholar%3Fq%3Dhead%2Bup%2Bdisplay%2BHUD%2Bsecond%2Blife%26btnG%3D%26hl%3Dfr%26as_sdt%3D0%252C5#search=%22head%20up%20display%20HUD%20second%20life%22

\cite{pmid17316094} %Northern study, sweden> Did like me, a software, but better

%Development of a PC-based diabetes simulator in collaboration with teenagers with type 1 diabetes.

%URL : http://online.liebertpub.com/doi/pdf/10.1089/dia.2006.0053
@article{nordfeldt2007development,
  title={Development of a PC-based diabetes simulator in collaboration with teenagers with type 1 diabetes},
  author={Nordfeldt, Sam and Hanberger, Lena and Malm, Fredrik and Ludvigsson, Johnny},
  journal={Diabetes technology \& therapeutics},
  volume={9},
  number={1},
  pages={17--25},
  year={2007},
  publisher={Mary Ann Liebert, Inc. 2 Madison Avenue Larchmont, NY 10538 USA}
}

% Other one : The Särimner diabetes simulator-a look in the rear view mirror

%URL http://online.liebertpub.com/doi/pdfplus/10.1089/dia.2006.0052
@article{hedbrant2007sarimner,
  title={The S{\"a}rimner diabetes simulator-a look in the rear view mirror},
  author={Hedbrant, Johan and Nordfeldt, Sam and Ludvigsson, Johnny},
  journal={Diabetes technology \& therapeutics},
  volume={9},
  number={1},
  pages={10--16},
  year={2007},
  publisher={Mary Ann Liebert, Inc. 2 Madison Avenue Larchmont, NY 10538 USA}
}


%Patient and Parent Views on a Web 2.0 Diabetes Portal—the Management Tool, the Generator, and the Gatekeeper: Qualitative Study
%URL http://www.ncbi.nlm.nih.gov/pmc/articles/PMC2956228/#!po=52.7778


%Diabetes Website Review: www.2aida.org
%http://online.liebertpub.com/doi/pdfplus/10.1089/dia.2005.7.741
@article{reed2005diabetes,
  title={Diabetes website review: www. 2aida. org},
  author={Reed, Karen and Lehmann, Eldon D},
  journal={Diabetes technology \& therapeutics},
  volume={7},
  number={5},
  pages={741--754},
  year={2005},
  publisher={Mary Ann Liebert, Inc. 2 Madison Avenue Larchmont, NY 10538 USA}
}