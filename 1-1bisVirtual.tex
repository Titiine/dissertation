\section{Virtual Worlds}

\subsection{Definition}
Defining virtual worlds is not easy. Since the early 80's, many definitions were given, closely linked with the technical evolutions in computer science~\cite{warburton2009second}. %Can be cited [warburton2009second] : check what i'm citing here ! 

\begin{figure}[h]
  \caption{A virtual auditorium in Second Life ~\cite{ref_VE_audit}}
  \centering
  \includegraphics[scale=0.6]{VE_auditorium.jpg}
  \label{fig:VE_audit}
\end{figure}

%Common recurrent patterns were identified, linked with synchronicity, persistance, avatar representation and network of users. 

One of the simplest approaches is to describe Virtual Environments in terms of a technological environment that make the user strongly feel that he is "present in an environment other than the one [he is] actually in", and compelled to "interact with that environment"~\cite{schroeder1996possible}. This kind of environment can find many usages, such as providing an environmement in order to realize tasks that cannot sometimes be made possible in real life, whether it's because the task cannot be done, given the existing physical constraints (flying, for instance). An historical, but still current use of virtual environments is to set classrooms or meetings, gathering virtually people that are in different places in real life. The virtual platfrom can provide a way to communicate, either by text, voice or video as shown in picture~\ref{fig:VE_audit}. The feeling of being present in the virtual environment is reinforced by being able to see exactly how many participants are sitting there, and even to interact with them, like shaking hands, or the ability to give other people virtual objects.

%[found in: Second Life in higher education: Assessing the potential for and the barriers to deploying virtual worlds in learning and teaching].


\subsection{A classification of virtual environments}
Multiplayer online games such as World of Warcraft were a major influence of modern virtual environments. Thus, commons features are shared, like the avatars that can be personalized as humanoid representations of the users in ~\ref{fig:VE_wow}.
Persistancy of the environment, interactions between multiple users, objects that can be owned and used, and an explorable world that presents similarities to the real world (physics, maps), enforcing the illusion of being in the game are other features commonly shared by this kind of games and virtual environments~\cite{warburton2009second}. 

\begin{figure}[h]
  \caption{World of Warcraft : Creation and personalization of an avatar~\cite{ref_VE_wow}}
  \centering
  \includegraphics[scale=0.7]{wow.jpg}
  \label{fig:VE_wow}
\end{figure}



\paragraph{}
Multiple classifications of Virtual environments were developped. A topology based on~\cite{mckeown2007taking} divided virtual environments in 4 main categories~\cite{warburton2009second}:
\begin{itemize}
\item Flexible narrative : Massively Multiplayer Online (MMO) games and serious games. Includes a scnearized evolution and goals to achieve.
\item Virtual social world : Social platforms, 3D chat rooms. Communication and socialization are especially emphasized, and there is no specific aim.
\item Simulation : simulations of the real world, such as Google Earth. Specific simulation needs are adressed, often for scientific or technical purposes. No inworld avatars, unless rare exceptions.
\item Workspace : 3D computer-supported workspaces. Commonly multiuser, with specific tools included according to the public aimed.

\end{itemize}

We will focus on multi-user virtual environments (MUVEs), which include the Social World and the Workspace categories. 
%MUVEs share many common features with massively multiplayers online games (MMOs) such as World of Warcraft. 


\subsection{Second Life}

Many virtual social worlds have an important community, among which Habbo, one of the oldest ones with 90 million users (Kzero Worldswide, 2009) and the famous Second Life, released in 2003~\cite{wang2009extending}. 
%Or cite this (guillet2013conducting): As of January 2010, over 162 million avatars have been registered to Habbo Hotel and there are an average 16.5 million unique visitors monthly (Sulake, n.d.).
\paragraph{}
It's is the most famous example of virtual environments, where avatars can explore the world by walking, flying and can create, use and sell objects. The example of second life make it clearer that virtual environments are not games, as there is no specific aim or storyline in-world. Socialization is a key issue, and every kind of real life activity can be done in-world, like swimming, dancing in parties or shopping in big malls created and owned by users.
  
%The kind of environment were are interested in Second Life, released in 2003~\cite{wang2009extending} is the most famous example of virtual environments, which . %
