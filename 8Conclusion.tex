\section{Project summary}
Diabetes educations is the key to improving diabetes management so that the risks of complications can be reduced. But diabetes education faces multiple challenges.\\
Firstly, a significant part of this education is done at the hospital when the patient is diagnosed. A huge amount of information is provided making it hard to remember afterward.
Secondly, understanding diabetes is mainly about experience : being able to understand and predict the complex changes in blood glucose, and teaching how to use the symptoms to match patient's feels with what is happening inside of his body.\\

This project focused on improving the quality of diabetes education, especially for adolescents. The solution investigated was to make it possible to simulate in real time the effects of diabetes so teenagers can learn by practise. It also provided them an area where all the 


\\In this dissertation, we analysed in the chapter 1 the basics of diabetes and diabetes-related topics, with behavioural, psychological and educational issues. The level of knowledge required was shown to be insufficient for an efficient diabetes management and to allow the patient to have a good quality of life. Different softwares for diabetics where studied, showing that the serious game option was an interesting one, but that the available software were inadequate for long term use because of a lack of attractiveness of content. Then, we investigated virtual environments, that appeared as a good option to run an educational simulation, focusing on Second Life and Opensimulator, the latter one being the ideal candidate to build our tool. To simulate a virtual diabetic avatar, we needed a model for the glucose-insulin system.
Bergman's model was too complex according to our needs, and didn't take into account long-acting insulin %(?or...) 
so we focused on the model used in Aida simulator.
In chapter 3, the design of the tool was described in term of needs, and we identified 3 use cases for the simulation. 
We created a suitable architecture for the application, which focuses on modularity (?) of the designed system, so that we can improve or change the model 
used for the glucose insulin system, and to add easily new objects and food items.
In chapter 4, the implementation of the software is explained. After a quick presentation of the tools used to build the simulation, 
we highlight how we built the different modules of the software, focusing on a problem-solving approach. 
Re-usability of the code was a key aspect, as shown with the animation library example and the way solutions were found for food and insulin as well.
Chapter 5 evaluate technically the tool and gives some opportunities for the future.

\section{Future work}

    \subsection{Closer interactions between the simulation and the educational content}
Connecting more the educational content and the simulation experience is the direction in which we are going. In fact, yesterday's educational software used by young teenagers and adolescents were usually rewarding the user when he studied a lesson and solved correctly the associated exercice. \\
The reward was either unlocking a new game that can be played in a specific ``resting area'', that had no link with the educational part of the software, or to win a new object that was needed to progress in the adventure, if the software consisted of an adventure that had no link with the educational content as well. In these two cases, the player considers that solving an exercice is a way to win new games or achieve an adventure.\\
Nowadays, games are so easy to find (for example, flash games everywhere on the internet) that it won't make sense to expect people to use the educational software to get items. Thus, a better approach is to make people really want to improve their level of knowledge about diabetes because they want to master what they consider as a game (the simulation). We should definitely focus on making the simulation appear as an MMO, where the aim is to improve the avatar. A strong focus should be put in the design of an appropriate system of levels with various parameters that evolve based on long term effects. For instance, the avatar could deal with welfare, physical condition, respect from its peers and popularity. These parameters could affect its appearance and what he is allowed to do in the simulation.

An icon displayed that shows the latest 
Tests displayed when

    \subsection{Future evolutions/impacts}
    "October 2011--The 3D Web is one step closer: Based on Unity3D, Jibe allows users to publish multiuser virtual worlds directly on the Web (to run in standard Web browsers) or on a mobile device http://bit.ly/AboutJibe - see also: Kataspace (plugin-free, HTML5 virtual worlds), Cloud Party (WebGL), Virtual World Framework (WebGL) and ProtonMedia ProtoSphere mobile (iPad) 3D virtual worlds"
    %from http://healthcybermap.org/sl.htm

%\section{Remerciements}
%    Uncomment 
    %A l'hopital, l'université, le superviseur
    %Ma famille, honey, mes amis