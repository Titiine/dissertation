%\section{Architecture}

%\subsection{Architecture of the software}
%TODO : explain how the different development tools from section ~\ref{sec:devtools} will be assembled.

%\subsection{Scripts main architecture - Overview}
\section{Scripts developed - Overview of the main architecture}
%TODO : Big schéma\\

This section describes the architecture of the application based on the different scripts and the links between them.
As OpenSimulator is used as a frame, there is some unusual constraint for coding: each script need to be attached to one object, and scripts cannot share data. \warn{For this reason, some objects were created only to host scripts inside.}\\

\begin{figure}[h!]
  \caption{Categories of scripts used in the simulation}
  \centering
  \includegraphics[scale=0.6]{globalScript.png}
  \label{fig:globalScript}
\end{figure}

Scripts can be divided in 3 categories as shown in figure~\ref{fig:globalScript}:
\begin{itemize}
\item The StateScript can be considered as the first category. Controlling the avatar's behaviour and its internal blood and insulin systems, it is the main script.
\item Scripts that perform a specific task strongly related to the nature of the object that is hosting the script. This category of domain-related scripts includes the FoodScript, the InsulinScript, and the MeterScript. All these scripts communicate with the StateScript, as described in ~\ref{ref:scriptComm}.
\item Service scripts, or scripts used as supports for other scripts or tasks. They can be considered as Pure Fabrications~\cite{neill2004imaging}. The maintenance scripts and the scripts used to display the Energy Bar and educational content about diabetes belong to this category.
\end{itemize}